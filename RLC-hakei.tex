\documentclass{ujarticle}

\begin{document}
\section{目的}
電気回路の構成素子であるR,L,C について,正弦波交流電圧を加えた場合の電流波形を観察する.また,RL 直列回路について特定の位相差を生じる条件を算出し,その波形を観察する.理論波形と観察波形を比較することにより,交流回路におけるL とC の働きを理解することを目的とする.

\section{L およびC の波形観察}
\subsection{R,L,C の電流特性}
\subsubsection{R の電圧電流特性}
抵抗R は無誘導性であるため,直流・交流ともに(\ref{ohm})式に示すオームの法則が成り立つ.
\begin{equation}
\label{ohm}
v = R・i
\end{equation}

ここで,v は抵抗に加える電圧,i は抵抗を流れる電流を意味する.波形観察の理解を進めるために電圧と電流を具体的な$v = V_{m}\sin (\omega t + \theta _{V} )$,$i = I_{m}\sin (\omega t + \theta _{I} )$という瞬時値で表すと(\ref{sin})式のようになる.
\begin{equation}
\label{sin}
V_{m}\sin (\omega t + \theta _{V} ) = RI_{m}\sin(\omega t + \theta _{I} )
\end{equation}

そして,(\ref{sin})式の振幅と位相を比較すると
\begin{equation}
\label{third}
V_{m} = RI_{m}
\end{equation}
\begin{equation}
\label{four}
\theta _{V} = \theta _{I}
\end{equation}
となり,電圧と電流の振幅は(\ref{third})式で表され,電圧と電流には位相の進みや遅れが生じない同相であることが分かる.

理想的な実験として,(\ref{sin})式に示される電圧と電流の波形を観察し,オームの法則が成り立つことを交流波形で観察することが望ましい.しかし,我々は電流波形を直接観測する装置を持たないため,交流電源とR のみの回路で電圧波形と電流波形を同時に観測することは容易ではない.その一方で,オシロスコープは測定したい素子の電圧波形を容易に観測することができる.以上の理由により,本実験では電源とR のみの回路の波形観測は行わない.その代わりに,L あるいはC の電流波形を観測するため,L あるいはC と直列に小さな値のR を接続する.そして,R の両端にオシロスコープの入力端子を接続し電圧波形をオームの法則から電流に変換して電流波形を求めるために用いる.

\subsubsection{L の電圧電流特性}
インダクタンスL は導線をらせん状に巻いたものであり,図1のように示される.このL に電源を接続すると,インダクタンスの内部に磁束$\phi$が生じる.インダクタンスL はこのように電気エネルギーを磁束として蓄えるため,電流は電圧よりも遅れる.この磁束$\phi$は時間 t と共に変化し,L の両端には磁束の変化を妨げる向きに電圧が生じる.
//
//
ここで,L に加わる電圧v,L に流れる電流i の関係は以下の式で表される.
\begin{equation}
\label{fifth}
v = L
\end{equation}

\end{document}