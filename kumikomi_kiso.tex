
\documentclass {ujarticle}
\usepackage[top=30truemm,bottom=30truemm,left=25truemm,right=25truemm]{geometry}

\title {組み込み制御技術の基礎}
\author{non name}
\date {2018年10月22日}
\begin {document}
\maketitle
\newpage

\section{目的}
炊飯器,テレビなどの身の回りの電子機器は,マイコン(マイクロ・コントローラー)を搭載し,特定の機能を実現させるためのプログラムが実装された「組み込みシステム」である.本実験では,Renesas 社製 RX63N シリーズ MCU(32bit))というマイコンを搭載した GR-SAKURA ボードを用い,制御プログラムの開発を通して,組み込み制御技術の基礎について理解する.

 \section{原理}
 GR-SAKURA ボードは,付録1に示す豊富な機能を内蔵している.また,C言語を用いたプログラムの作成には,クラウドを利用した開発環境(webコンパイラ)が提供されていることから,web 接続が可能な端末さえあればプログラミングが可能となっている.
 
 GR-SAKURA ボードには,豊富な入出力端子が用意されているが,今回の実験では,デジタル出力,デジタル入力,および,アナログ入力の利用方法について学ぶ.デジタル出力については,光として視覚に訴える発光ダイオードと,音として聴覚に訴える圧電ブザーを,デジタル入力についてはプッシュスイッチを,アナログ入力には受光面の明るさに応じて抵抗値が変化する CdS を接続し,プログラムからこれらにアクセス/制御する.
 
 GR-SAKURA ボードには,USB ケーブルを介して 5V が給電されているが,今回のセッティングでは外部接続用の入出力コネクタの動作レベルは 3.3V となるため,発光ダイオードには 100Ω の保護抵抗を,CdS には 10kΩ の分圧抵抗を設ける.
 
 \section{実験方法}
 \subsection{使用器具}
 本実験では以下の器具を使用した.
 \begin{enumerate}
 \item GR-SAKURA(No.8)
 \item PC(IEM-108)
 \item テスター
 \item パーツセット(No.8)「LED 8個,100Ω抵抗 8個,10kΩ抵抗 1個,圧電ブザー,CdS,CdSカバー,ジャンパー線,ブレッドボード,mini-USB ケーブル」
 \end{enumerate}
 
 \subsection{制御対象となる外部接続回路の作成}
 前述の素子を用いた制御対象となる回路を図1に示す.GR-SAKURA を図2に示す.GR-SAKURA ボードの外部接続用の入出力コネクタには限りがあるため,制御対象となる回路はブレッドボード上に作成し,ジャンパー線を用いて GR-SAKURA ボードの各ポートと接続する.このとき,GR-SAKURA ボード上のプッシュスイッチへ容易にアクセスできるように,ジャンパー線の長さや取り回しを工夫すること.なお,プッシュスイッチについては,GR-SAKURA ボード上に実装されている青色のスイッチを流用する.
 
 \subsection{web コンパイラへのログインとサンプルの動作確認}
 
 
\end{document}
