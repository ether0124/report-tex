
\documentclass {ujarticle}
\usepackage[top=30truemm,bottom=30truemm,left=25truemm,right=25truemm]{geometry}
\usepackage{listings,jlisting}
\usepackage[dvipdfmx]{graphicx}
\usepackage{indentfirst}

\lstset{%
  language={C},
  basicstyle={\small},%
  identifierstyle={\small},%
  commentstyle={\small\itshape},%
  keywordstyle={\small\bfseries},%
  ndkeywordstyle={\small},%
  stringstyle={\small\ttfamily},
  frame={tb},
  breaklines=true,
  columns=[l]{fullflexible},%
  numbers=left,%
  xrightmargin=0zw,%
  xleftmargin=3zw,%
  numberstyle={\scriptsize},%
  stepnumber=1,
  numbersep=1zw,%
  lineskip=-0.5ex%
}

\title {組み込み制御技術の基礎}
\author{non name}
\date {2018年10月22日}
\begin {document}
\maketitle
\newpage

\section{目的}
炊飯器,テレビなどの身の回りの電子機器は,マイコン(マイクロ・コントローラー)を搭載し,特定の機能を実現させるためのプログラムが実装された「組み込みシステム」である.本実験では,Renesas 社製 RX63N シリーズ MCU(32bit))というマイコンを搭載した GR-SAKURA ボードを用い,制御プログラムの開発を通して,組み込み制御技術の基礎について理解する.

 \section{原理}
 GR-SAKURA ボードは,付録1に示す豊富な機能を内蔵している.また,C言語を用いたプログラムの作成には,クラウドを利用した開発環境(webコンパイラ)が提供されていることから,web 接続が可能な端末さえあればプログラミングが可能となっている.
 
 GR-SAKURA ボードには,豊富な入出力端子が用意されているが,今回の実験では,デジタル出力,デジタル入力,および,アナログ入力の利用方法について学ぶ.デジタル出力については,光として視覚に訴える発光ダイオードと,音として聴覚に訴える圧電ブザーを,デジタル入力についてはプッシュスイッチを,アナログ入力には受光面の明るさに応じて抵抗値が変化する CdS を接続し,プログラムからこれらにアクセス/制御する.
 
 GR-SAKURA ボードには,USB ケーブルを介して 5V が給電されているが,今回のセッティングでは外部接続用の入出力コネクタの動作レベルは 3.3V となるため,発光ダイオードには 100Ω の保護抵抗を,CdS には 10kΩ の分圧抵抗を設ける.
 
 \section{実験方法}
 \subsection{使用器具}
 本実験では以下の器具を使用した.
 \begin{enumerate}
 \item GR-SAKURA(No.8)
 \item PC(IEM-108)
 \item テスター
 \item パーツセット(No.8)「LED 8個,100Ω抵抗 8個,10kΩ抵抗 1個,圧電ブザー,CdS,CdSカバー,ジャンパー線,ブレッドボード,mini-USB ケーブル」
 \end{enumerate}
 
 \subsection{制御対象となる外部接続回路の作成}
 前述の素子を用いた制御対象となる回路を図1に示す.GR-SAKURA を図2に示す.GR-SAKURA ボードの外部接続用の入出力コネクタには限りがあるため,制御対象となる回路はブレッドボード上に作成し,ジャンパー線を用いて GR-SAKURA ボードの各ポートと接続する.このとき,GR-SAKURA ボード上のプッシュスイッチへ容易にアクセスできるように,ジャンパー線の長さや取り回しを工夫すること.なお,プッシュスイッチについては,GR-SAKURA ボード上に実装されている青色のスイッチを流用する.
 
 \subsection{web コンパイラへのログインとサンプルの動作確認}
 外部接続回路の準備ができたら,パソコンを用意して web ブラウザを起動し,Renesas 社が提供する「がじぇっとるねさす」サイト http://gadget.renesas.com/ja/ を表示し,画面の上部中央に用意された「ゲストログイン」を選択する. 
 
 web コンパイラの利用上の注意事項,および,ゲストログインに関する注意事項について同意もしくは承諾すると,テンプレート選択が促されるので,今回のターゲットである※「 GR-SAKURA\_Sketch V2.20.zip」を選択し,適切なプロジェクト名を付与してプロジェクトを作成する.\\※第1,2回目実験時はV1.08を使用していたが削除されてしまったため第3回目実験時のみV2.20を使用した.また,テンプレートとして,GR-KURUMI や GR-KAEDE は使用しないこと.
 
 前項の操作により,GR-SAKURA の開発環境である Renesas Web Compiler が表示されるので,左側のプロジェクトイベントに表示されている「gr\_sketch.cpp」をダブルクリックしてソースコードを表示し,その内容を確認する.
 
 続いて,ウィンドウ右部のメニューペインから「ビルド実行」を選択してコンパイルおよびリンクを行う.なお,プログラムを修正した場合など,必要に応じて,ビルド実行の前に「保存」を選択すると,クライアントとサーバーのソースコードを強制的に同期させることができる.
 
 ビルドに成功すると,左側のプロジェクトペインにバイナリモジュール「sketch.bin」が作成・更新されているので,これをマウスの左ボタンでクリックして表示されるメニューから「ダウンロード」を選択して,自分のパソコンにダウンロードする.このとき,ダウンロードしバイナリモジュールの保存先(フォルダ)をしっかりと記憶すること.
 
 パソコンと GR-SAKURA ボードを USB ケーブルで接続し,マイコンピュータを開いて GR-SAKURA ボードがリムーバルメディアとして認識されていない場合,GR-SAKURA ボードの赤いプッシュスイッチを押して,動作状態を初期化すると復旧(認識)される.
 
 リムーバルメディアとして認識された GR-SAKURA ボードのトップフォルダ(ルートフォルダ)に,先ほどダウンロードしたバイナリモジュール( sketch.bin )を転送すると,GR-SAKURA ボードが自動的にプログラム実行モードで再起動するので,プログラムが正しく動作していることを確認する.
 \subsection{使用したプログラム}
 本実験では5種類のプログラムを使用した.
 \subsubsection{プッシュスイッチ}
 \begin{lstlisting}
#include<rxduino.h>

#define INTERVAL	50
#define PIN_IO0		0

void setup()
{
	for(int i = 0; i <= 8; i++){
		pinMode(PIN_IO0 + i, OUTPUT);
	}
	pinMode(PIN_SW,INPUT);
}

void loop()
{
	if(digitalRead(PIN_SW) == 0){
		for(int i=0; i < 8; i++){
			digitalWrite(PIN_IO0 + i,1);
			delay(INTERVAL);
		}
		for(int i=0; i < 8; i++){
			digitalWrite(PIN_IO0 + i,0);
			delay(INTERVAL);
		}
	}else{
		for(int i =0; i < 8; i++){
			digitalWrite(PIN_IO0 + i,0);
		}
	}
}

 \end{lstlisting}
 
\bigskip
 以下,4つのプログラムはパソコンとシリアル通信をする必要がある. プログラム実行後パソコンで TeraTerm を開き,GR-SAKURA と接続しているポートを選択する.キーボードのいずれかのキーを押すとコンソール画面に「Connect!」と表示され接続が完了し,シリアル通信ができるようになる.
 \subsubsection{CdS入力}
 \begin{lstlisting}
#include<rxduino.h>

#define INTERVAL	100
#define	PIN_IO0		0
#define PIN_AN0		14

void setup()
{
	for(int i = 0; i <= 8; i++){
		pinMode(PIN_IO0 + i,OUTPUT);
	}
	analogReference(EXTERNAL);
	Serial.begin(9600);
	while(Serial.available() == 0) { ; }
	Serial.println("Connect!");
}

void loop()
{
	unsigned int cds = analogRead(PIN_AN0);
	Serial.print("CdS = ");
	Serial.println(cds);
	unsigned int level = 100;
	for(int i = 0; i < 8; i++){
		if(cds < level) digitalWrite(PIN_IO0 + i,1);
		else digitalWrite(PIN_IO0 + i,0);
		level += 100;
	}
	delay(INTERVAL);
}

 \end{lstlisting}
 
 \subsubsection{PWM制御}
 \begin{lstlisting}
#include<rxduino.h>

#define INTERVAL	100
#define	PIN_IO0		0
#define PIN_AN0		14

void setup()
{
	for(int i = 0; i <= 8; i++){
		pinMode(PIN_IO0 + i,OUTPUT);
	}
	analogReference(EXTERNAL);
	Serial.begin(9600);
	while(Serial.available() == 0);
	Serial.println("Connect!");
}

void loop()
{
	unsigned int cds = analogRead(PIN_AN0);
	Serial.print("CdS = ");
	Serial.println(cds);	
	int level = 255 - cds / 4;
	for (int i = 0; i < 8; i++){
		analogWrite(PIN_IO0 + i,level);
	}
	delay(INTERVAL);
}
 \end{lstlisting}
 
 \subsubsection{圧電ブザー}
 \begin{lstlisting}
#include<rxduino.h>

#define INTERVAL	100
#define	PIN_IO8		8
#define PIN_AN0		14

void setup()
{
	pinMode(PIN_IO8, OUTPUT);
	analogReference(EXTERNAL);
	Serial.begin(9600);
	while(Serial.available() == 0);
	Serial.println("Connect!");
}

void loop()
{
	unsigned int cds = analogRead(PIN_AN0);
	Serial.print("CdS = ");
	Serial.println(cds);
	int frequency = 3300 -cds * 3;
	tone(PIN_IO8, frequency, 20);
	delay(INTERVAL);
}

 \end{lstlisting}

 \subsubsection{シングルクリック}
 \begin{lstlisting}
#include<Arduino.h>

#define PIN_IO0		0
#define	SWITCH_PUSH	0	
#define LED_ON		1
#define LED_OFF		0

int position = 0;
unsigned long now = 0;
unsigned long action = 0;

void setup()
{
	for(int i=0; i < 8; i++){
		pinMode(PIN_IO0 + i, OUTPUT);
		digitalWrite(PIN_IO0 + i,LED_OFF);
	}
	Serial.begin(9600);
	while(Serial.available() == 0);
	Serial.println("Connect!");
	digitalWrite(position, LED_ON);
}

void loop()
{
	now =millis();
	if (digitalRead(PIN_SW) == SWITCH_PUSH) {
		if(action == 0) action = now;
	}else{
		if(action > 0){
			digitalWrite(position, LED_OFF);
			position++;
			if(position >= 8) position = 0;
			digitalWrite(position, LED_ON);
			Serial.println(now - action);
			action = 0;
		}
	}
}

 \end{lstlisting}
 
 \section{実験結果}
 実験方法3.4 に記載したプログラムを GR-SAKURA 上で実行した際の外部接続回路,またはコンソール画面の様子を実験結果としてまとめる.
 \subsection{プッシュスイッチ}
 プログラム実行後,GR-SAKURA 上のプッシュスイッチをクリック,ダブルクリックしたとき,プログラムから PIN\_IO につないだ LED に点灯または消灯の命令が入力され,出力として PIN\_IO に繋がれた8本の LED が一定の時間間隔で端から順に点灯し,その後同じ順で消灯をした.また,プッシュスイッチを長押ししている間はこの動作を繰り返した.逆に,スイッチを押していないときは,プログラムから LED に消灯の命令の入力が繰り返され,LED は消灯を続けた.
 
\subsection{CdS入力}
プログラム実行後,LED が点灯し,コンソール画面に CdS の値が連続的に出力された.CdS に当たる光の量をキャップなどを使って物理的に操作すると,暗くしたときCdS の値は大きくなり LED の点灯数が減少した.明るくすると値は小さくなり,LED の点灯数が増えた.またこのとき,プログラムに代入された CdS の値が変化したためプログラムの流れが変化し,結果として出力先である LED の動作が変わった事が分かった.

\subsection{PWM制御}
 プログラム実行後,すべての LED が点滅し,コンソール画面に CdS の値が連続的に出力された.4.2 と同様にして,入力となっている CdS を物理的に操作すると,強い光を当てたとき出力である LED も初期よりも強い光を発した.また暗くしたとき,LED の光も初期に比べ弱くなる事が分かった.
 
 \subsection{圧電ブザー}
 プログラム実行後.圧電ブザー(出力)が一定のリズムで鳴り,コンソール画面に CdS の値が連続的に出力された.4.2 と同様にして,入力となっている CdS を物理的に操作した.強い光を当てたとき,ブザーは初期よりも高いを音を鳴らし,逆に暗くしたとき,ブザーはより低い音を鳴らした.また,プログラム中の変数「frequency」に代入している式の数値を変えてみるとブザーの音の高さが変化したため,出力も変化した事が分かった.
 
 \subsection{シングルクリック}
 プログラム実行後,出力であるLED が一つ点灯した.GR-SAKURA 上のプッシュスイッチをクリックすると,LED の点灯位置が一つずれ,コンソール画面にLEDスイッチを押した時刻と離した瞬間の時刻の時間差が出力された.また,プッシュスイッチを長押ししたときにも同じ動作をした.そしてチャタリングという,クリック時と同じ動作を高速で数回繰り返す現象を3回に1回程度の頻度で観測する事ができた.このプログラムではプッシュスイッチを入力として,押すとLED の位置がクリックするごとに一つずつずれていくというプログラムとなっており,前述のような動作は入力と出力が合っておらず不具合といえる.詳細は課題5にて記述する.
 
 \section{課題}

 \begin{enumerate}
 \item LED の順方向電圧を 1.9[V]と仮定した場合,LED 点灯時に流れる電流を求めなさい.
 
 \item 図1の外部接続回路において,アナログ入力ポート(AN0)で検出かのな入力電圧の最小値と最大値,および,マイコン内部に取り込まれた入力値の最小値と最大値はいくつか.
 
 アナログ入力ポート(AN0)で検出可能な入力電圧の最小値と最大値は,それぞれ0V,3.3V である.またマイコン内部に取り込まれた入力値の最小値と最大値は,それぞれ0,1023 である.
 
 \item プログラム3.4.4圧電ブザー において,最も暗くしたときに再生される音の周波数,および,最も明るくしたときに再生される音の周波数はいくつか.
 
  最も暗くしたとき再生される音の周波数は231[hz]になる.また,最も明るくしたとき再生される音の周波数は3300[hz]になる.
 
 \item プログラム3.4.5シングルクリック において,シリアル通信を介してパソコンに表示される数値は何を表しているか.
 
  シリアル通信を介してパソコンに表示される数値は,スイッチを離した瞬間の時刻とスイッチを押した時刻の時間差を表している.
 $[スイッチを離した時刻T_{2} - スイッチを押した時刻T_{1}]$
 
 \item プログラム3.4.5シングルクリック には,再現性のある不具合が含まれている.この不具合を発見,および,原因を特定し,プログラムを修正しなさい.また,修正した内容について,プログラムやフローチャートなどを示し,工夫した点を述べなさい.
 
  この不具合の原因はチャタリングというごく短時間にスイッチが高速に押される現象であるという事が分かった.これはハードウェア的な問題であり,プログラムを改良することによってある程度対処が可能である.その改良後プログラムを下に記述する.
 
 \begin{lstlisting}
 #include<Arduino.h>

#define PIN_IO0		0
#define	SWITCH_PUSH	0
#define LED_ON		1
#define LED_OFF		0

int position = 0;
unsigned long now = 0;
unsigned long action = 0;
int chatime = 0;

void setup()
{
	for(int i=0; i < 8; i++){
		pinMode(PIN_IO0 + i, OUTPUT);
		digitalWrite(PIN_IO0 + i,LED_OFF);
	}
	Serial.begin(9600);
	while(Serial.available() == 0);
	Serial.println("Connect!");
	digitalWrite(position, LED_ON);
}

void loop()
{
	now =millis();

	if (digitalRead(PIN_SW) == SWITCH_PUSH ){
		if(action == 0) action = now;
	}else{
	    chatime = now - action;

	    if(action >0 && chatime >20){
		digitalWrite(position, LED_OFF);
		position++;
		if(position >= 8) position = 0;
		digitalWrite(position, LED_ON);
		Serial.println(now - action); 
		action = 0;

	    }else{
	        action =now;
	    }
	}
}
 \end{lstlisting}
 
  このプログラムでは,チャタリングとクリックを判別し,チャタリングが起こったと思われる場合は,LED についての処理を何もせず終わるようにしている.判定には,クリックの押してから離すまでの時間差を利用している.一般的にチャタリングの場合,時間差は10〜20msとされている.人間ではこのような高速なクリックは不可能と考え,チャタリングとして判定している.
 \end{enumerate}
 
 \section{研究事項}
 
 \begin{enumerate}
 
 \item CdS(硫化カドミウム)の電気特性について調べ,まとめなさい.
 
 
 「焼結形硫化カドミウム系光導電セルは,光量によりその内部抵抗の変化する可変抵抗素子で,ある照度L lx の光を被照させ,電圧Vvoltを印加し たとき,その光電流Iamp は一般に次式のように表わされる.
\[
I\propto(e\mu\tau|l^{2})V^{\alpha}・L^{\beta}
\]

 ここでαは電圧に対する指数,β は照度に対する指数である.す なわち光電流は照度,電 圧に依存するだけでな く,先 の利得のところで述べたように光導電物質,とりわけ不純物の種類・濃度,製 造条により変化することおよび電極間隔の二乗に逆比例することを示している.
 
 電圧に対する指数αは,通常1.0~1.2ぐらいであり,光電流は一定照度に対しては印加電圧にほとんど比例して変化する.

 照度に対する指数βは,不純物など原材料の配合・組成および焼結条件などによって異なるが,通常0.6~ 1.0ぐらいである.一 般にこの値は低照度で高く高照度で低い.
 第2図は光導電セルの照度一印加電圧一光電流特性 の一例を示すが,以上に述べたこととよく合致している.」”光導電素子の原理と特性”より引用

\begin{center}
\includegraphics[width=0.6\textwidth]{2018-10-28-1-25-57.png}
\end{center}

 \item PWM(パルス幅変調)について調べ,図解しなさい. 
 
   「PWM(Pulse Width Modulation)とは、半導体を使った電力を制御する方式の1つです。オンとオフの繰り返しスイッチングを行い、出力される電力を制御します。

 一定電圧の入力から、パルス列のオンとオフの一定周期を作り、オンの時間幅を変化させる電力制御方式を PWM と呼びます。早い周期でスイッチングを行うことで、オンのパルス幅に比例した任意の電圧が得られます。これは、半導体がオンとオフ状態が最も損失が少ない(中間状態は損失多い)ことを利用した電力制御方式です。

 PWMは、優れた制御性と、高効率 が特長で、インバータ回路 で広く使われている技術です。ブラシ付きDCモータの回転制御にも使われています。 

 インバータ回路にて、PWM制御のオンの時間幅(デューティ)を周期的に変化させることにより、モータ駆動に最適な正弦波の交流電圧を作ることができます。」”TOSHIBA 半導体&ストレージ製品 >・・・> 第3章:ブラシレスモーターの技術説明 > PWMとは https://toshiba.semicon-storage.com/jp/design-support/e-learning/brushless\_motor/chap3/1274512.html ” より引用 
 
 \begin{center}
 \includegraphics[width=0.7\textwidth]{P22-01.jpeg}
 
 図3 PWM の説明図
 \end{center}
 
 \end{enumerate}
 \section{考察}
 
 \section{参考文献}
 \begin{enumerate}
 \item 矢野昌平 監,ものづくり技術実習II 2年(後期)長岡高専電気電子システム工学科 
 \item 光導電素子の原理と特性,https://www.jstage.jst.go.jp/article/jieij1917/47/11/47\_11\_540/\_pdf
 \item TOSHIBA 半導体&ストレージ製品 >・・・> 第3章:ブラシレスモーターの技術説明 > PWMとは,https://toshiba.semicon-storage.com/jp/design-support/e-learning/brushless\_motor/chap3/1274512.html 
  \end{enumerate}
 
 
 
 
 
 
 
 
 
 
 
 
 
 
 
 
 
 
 
 
 
 
 
\end{document}
