\documentclass{ujarticle}
\usepackage[top=30truemm,bottom=30truemm,left=25truemm,right=25truemm]{geometry}
\usepackage{listings,jlisting}
\usepackage[dvipdfmx]{graphicx}
\usepackage{indentfirst}

\lstset{%
  language={C},
  basicstyle={\small},%
  identifierstyle={\small},%
  commentstyle={\small\itshape},%
  keywordstyle={\small\bfseries},%
  ndkeywordstyle={\small},%
  stringstyle={\small\ttfamily},
  frame={tb},
  breaklines=true,
  columns=[l]{fullflexible},%
  numbers=left,%
  xrightmargin=0zw,%
  xleftmargin=3zw,%
  numberstyle={\scriptsize},%
  stepnumber=1,
  numbersep=1zw,%
  lineskip=-0.5ex%
}

\begin{document}
\section{目的}
2つの増幅器(トランジスタ)等とコンデンサと抵抗をたすき掛け形に接続すると,ONとOFFを繰り返す発振器を作製することができる.本実験では,実際にユニバーサル基板上にはんだごてを用いて,マルチバイブレータ回路を作製し,LEDの点灯が繰り返されることを確認する.また,回路の動作原理を理解し,点灯時間の制御方法について理解する.

\section{理論}
\subsection{マルチバイブレータ回路(multivibrator circuit)}
図1にマルチバイブレータ回路図を示す.この回路では,一方のトランジスタがONとなり,もう一方がOFFとなる.コンデンサの充電・放電によって,トランジスタのONとOFFは交互に繰り返される.

\subsection{マルチバイブレータの動作原理}
状態1) 初期状態で,トランジスタQ2がONと仮定する.初期状態で,LED2は点灯している.反対にトランジスタQ1はOFFとなり,LED1は消灯している.コンデンサC1は電源EからLED1と抵抗R1を通し流れた電流によって充電(電荷がたまる)される.

状態1では,LED1に電流が流れているように見えるが,コンデンサC1への充電はすぐに終了し,この電流は流れなくなる.そのためLED1は瞬時に消灯する.

コンデンサC1の電位を$V_{c}$とする.\\
\\
 状態2) コンデンサC2は電源Eから抵抗R3を通し流れた電流によって充電(電荷がたまっている)される.この充電時間は,コンデンサC2の容量と抵抗R3,充電前の状態とによって決定される.

R3はR1に比べ抵抗値が大きいので,充電時間はコンデンサC1よりも長くなる.(C2の充電前は,マイナスの電位を持っている状態である.状態4を参照すること.)\\
\\
 状態3) コンデンサC2への充電が進みC2の左側の電位が上がってくる.そしてトランジスタQ1のベース・エミッタ間電圧VBE=0.6~0.7Vよりも高くなるとトランジスタQ1のベースに電流が流れトランジスタQ1がONとなる.\\
\\
 状態4) トランジスタQ1がONになると,コンデンサC1の+側の電位はGND(0V)となる.
\\
コンデンサC1の+側の端子がGNDレベルとなったことで,コンデンサC1反対側の電位は,マイナスの電位となる.コンデンサC1には状態1により$V_{c}$の電位が充電されていたため,コンデンサC1の-側の端子電位は-$V_{c}$となる.

これにより,トランジスタQ2のベースの電位が下がりトラジスタQ2はOFFとなる.

以降は,素子は逆になるが状態1~4が繰り返される.

\end{document}